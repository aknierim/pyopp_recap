\secslide{Testing}

\begin{frame}{When Do We Need Tests?}
  \begin{center}
    \huge\textcolor{ccyan!90!cblack}{Imagine the following\dots}
  \end{center}
  \begin{itemize}
    \item You have written a package with a lot of code, \eg, multiple scripts
    \item You found a bug somewhere in your code
    \item You have not thought of possible edge cases during development
  \end{itemize}
  \vspace{1em}
  \begin{center}
    \huge\textcolor{cpink!90!cblack}{\to{} You will need to investigate your codebase for
    causes of the bug and even then the same bug may appear some time later}
  \end{center}
\end{frame}

\begin{frame}{Solution}
  \begin{center}
    \huge\textcolor{ccyan!90!cblack}{Write persistent tests \textbf{during development!}}\\
    \uncover<2>{\huge\textcolor{ccyan!90!cblack}{(And \textbf{automate} them \to{} see CI)}}

    \begin{tikzpicture}[remember picture, overlay]
      \only<2>{
        \node [anchor=south] at ([yshift=1cm]current page.south) {
          \includegraphics[height=3cm]{graphics/automation_meme.png}
        };
      }
    \end{tikzpicture}
  \end{center}
\end{frame}

\begin{frame}{Test Levels}
  \begin{description}[Operational Acceptance Testing]
    \setlength{\itemsep}{1em}
    \item [Unit Testing] Test single units (\ie, single functions or classes) of your software.
    \item [Integration Testing] Test multiple components that depend on each other.
    \item [System Testing] Test the entire software with respect to its requirements, \eg, I/O data.
    \item [Operational Acceptance Testing] Give your software to the user to break it.
  \end{description}
  \vspace{0.5cm}
  \uncover<2>{
    \begin{center}
      \includegraphics[height=3cm]{graphics/onedoesnot.jpg}
    \end{center}
  }
\end{frame}

\begin{frame}{What Do We Test For?}
  \begin{center}
    \huge\textcolor{ccyan}{This is probably the hardest part\dots}
  \end{center}
  \begin{itemize}
    \setlength{\itemsep}{1em}
    \item You will need to understand your code
    \item You will need to verify how much and what parts of your code are covered by tests
    \item Even then your code may not be guaranteed to work error-free
    \item \emph{Good practice}: Every time you find a bug, add a unit test so it doesn't reappear
  \end{itemize}
\end{frame}

\begin{frame}{Tools}
  Shipped with Python:
  \begin{description}
    \item [\iref{https://docs.python.org/3/library/doctest.html}{doctest}] Allows you to write simple tests in the docstrings
      of your functions.
    \item [\iref{https://docs.python.org/3/library/unittest.html}{unittest}] Allows you to write regular unit tests,
      \ie, separate functions and classes that test your code.
  \end{description}
  \vspace{0.5cm}
  Further tools:
  \begin{description}
    \item [\iref{https://docs.pytest.org/en/stable/}{pytest}] Scalable, and easy to use test framework.
  \end{description}
\end{frame}
