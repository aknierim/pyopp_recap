\secslide{Testing}

\begin{frame}{When Do We Need Tests?}
  \begin{center}
    \huge\textcolor{ccyan!90!cblack}{Imagine the following\dots}
  \end{center}
  \begin{itemize}
    \item You have written a package with a lot of code, \eg, multiple scripts
    \item You found a bug somewhere in your code
    \item You have not thought of possible edge cases during development
  \end{itemize}
  \vspace{1em}
  \begin{center}
    \huge\textcolor{cpink!90!cblack}{\to{} You will need to investigate your codebase for
    causes of the bug and even then the same bug may appear some time later}
  \end{center}
\end{frame}

\begin{frame}{Solution}
  \begin{center}
    \huge\textcolor{ccyan!90!cblack}{Write persistent tests \textbf{during development!}}\\
    \uncover<2>{\huge\textcolor{ccyan!90!cblack}{(And \textbf{automate} them \to{} see CI)}}

    \begin{tikzpicture}[remember picture, overlay]
      \only<2>{
        \node [anchor=south] at ([yshift=1cm]current page.south) {
          \includegraphics[height=3cm]{graphics/automation_meme.png}
        };
      }
    \end{tikzpicture}
  \end{center}
\end{frame}

\begin{frame}{Test Levels}
  \begin{description}[Operational Acceptance Testing]
    \setlength{\itemsep}{1em}
    \item [Unit Testing] Test single units (\ie, single functions or classes) of your software.
    \item [Integration Testing] Test multiple components that depend on each other.
    \item [System Testing] Test the entire software with respect to its requirements, \eg, I/O data.
    \item [Operational Acceptance Testing] Give your software to the user to break it.
  \end{description}
  \vspace{0.5cm}
  \uncover<2>{
    \begin{center}
      \includegraphics[height=3cm]{graphics/onedoesnot.jpg}
    \end{center}
  }
\end{frame}

\begin{frame}{What Do We Test For?}
  \begin{center}
    \huge\textcolor{ccyan}{This is probably the hardest part\dots}
  \end{center}
  \begin{itemize}
    \setlength{\itemsep}{1em}
    \item You will need to understand your code
    \item You will need to verify how much and what parts of your code are covered by tests
    \item Even then your code may not be guaranteed to work error-free
    \item \emph{Good practice}: Every time you find a bug, add a unit test so it doesn't reappear
  \end{itemize}
\end{frame}

{
  \usemintedstyle{code-light}
\begin{frame}[fragile]{Tools}
  Shipped with Python:
  \begin{description}
    \item [\iref{https://docs.python.org/3/library/doctest.html}{doctest}] Allows you to write simple tests in the docstrings
      of your functions.
    \item [\iref{https://docs.python.org/3/library/unittest.html}{unittest}] Allows you to write regular unit tests,
      \ie, separate functions and classes that test your code.
  \end{description}
  \vspace{0.5cm}

  Additional tools:
  \begin{description}[\iref{https://coverage.readthedocs.io/en/7.9.2/}{Coverage.py}]
    \item [\iref{https://docs.pytest.org/en/stable/}{pytest}] Scalable, extensible (\ie, through plugins), and easy to use test framework.
    \item [\iref{https://coverage.readthedocs.io/en/7.9.2/}{Coverage.py}] A tool for measuring code coverage. Works well with \texttt{pytest}
      if \iref{https://pytest-cov.readthedocs.io/en/latest/}{pytest-cov} is installed:
      \begin{minted}{shell-session}
        $ pytest --cov
      \end{minted}
    \item [\iref{https://tox.wiki/en/4.28.1/}{tox}] A generic virtual environment management and test command line tool. Can be used to:
      \begin{itemize}
        \item [\to] Check whether your package builds and installs in different envs
        \item [\to] Run tests in each defined env, \eg, using \texttt{pytest}
      \end{itemize}
    \item [\iref{https://nox.thea.codes/en/stable/}{Nox}] Similar to \texttt{tox}, but uses standard Python files and decorators for configuration.
      (For differences, see \iref{https://hynek.me/articles/why-i-like-nox/}{Why I Like Nox} by Hynek Schlawack)
  \end{description}
\end{frame}
}

\begin{darkframe}[fragile]{pytest}
  \begin{center}
    \large\textcolor{ccyan}{An example taken from the \texttt{pytest} docs}
  \end{center}
  \vspace{0.25cm}
  \begin{columns}[t,onlytextwidth]
    \begin{column}{0.28\textwidth}
      \begin{minted}{python}
        # test_sample.py
        def inc(x):
            return x + 1


        def test_answer():
            assert inc(3) == 5
      \end{minted}
    \end{column}
    \hfill
    \begin{column}{0.68\textwidth}
      \footnotesize
      % oh boi, the following caused me pain
      \begin{minted}[escapeinside=||]{shell-session}
        $ pytest test_sample.py
        =========================== test session starts ============================
        platform linux -- Python 3.x.y, pytest-8.x.y, pluggy-1.x.y
        rootdir: /home/sweet/project
        collected 1 item

        test_sample.py |\textcolor{cred}{F}|                                                     |\textcolor{cred}{[100\%]}|

        ================================= FAILURES =================================
        |\textcolor{cred}{\_\_\_\_\_\_\_\_\_\_\_\_\_\_\_\_\_\_\_\_\_\_\_\_\_\_\_\_\_\_\_ test\_answer \_\_\_\_\_\_\_\_\_\_\_\_\_\_\_\_\_\_\_\_\_\_\_\_\_\_\_\_\_\_\_\_}|

            def test_answer():
        >       assert inc(3) == 5
        |\textcolor{cred}{E}|       |\textcolor{cred}{assert 4 == 5}|
        |\textcolor{cred}{E}|        |\textcolor{cred}{+  where 4 = inc(3)}|

        |\textcolor{cred}{test\_sample.py}|:6: AssertionError
        |\textcolor{ccyan}{========================= short test summary info ==========================}|
        |\textcolor{cred}{FAILED}| test_sample.py::test_answer - assert 4 == 5
        |\textcolor{cred}{============================ 1 failed in 0.12s =============================}|
      \end{minted}
    \end{column}
  \end{columns}
\end{darkframe}

\begin{splitframe}[fragile]{pytest}{pyproject.toml}
  \begin{columns}[t,onlytextwidth]
    \begin{column}{0.58\textwidth}
      \begin{itemize}
        \setlength{\itemsep}{1em}
        \item \texttt{pytest} runs all functions starting with \texttt{test\_} (or classes starting with \texttt{Test}).
        \item Provide \texttt{pytest} with the file that contains the specific test functions you want to run.
        \item If no arguments are provided to \texttt{pytest}, it looks for paths defined in \texttt{testpath} (if defined)
          \begin{itemize}
            \item [\to] Otherwise: Recursive search for files matching \texttt{test\_*.py} or \texttt{*\_test.py}
          \end{itemize}
        \item Prints are suppressed per default; use the \texttt{-s} flag to see prints:
          {
          \usemintedstyle{code-light}
          \begin{minted}{shell-session}
            $ pytest -s
          \end{minted}
          }
      \end{itemize}
    \end{column}
    \begin{column}{0.38\textwidth}
      \begin{minted}{toml}
        [tool.pytest.ini_options]
        testpaths = [
          "tests",
        ]
        addopts = "--verbose"
      \end{minted}
    \end{column}
  \end{columns}
\end{splitframe}

\begin{darkframe}[fragile]{Useful Feature: pytest Fixtures}
  \begin{columns}[t,onlytextwidth]
    \begin{column}{0.38\textwidth}
      \small
      \begin{minted}{python}
        # contents of test_append.py (pytest)
        import pytest


        # Arrange
        @pytest.fixture
        def first_entry():
            return "a"


        # Arrange
        @pytest.fixture
        def order(first_entry):
            return [first_entry]


        def test_string(order):
            # Act
            order.append("b")

            # Assert
            assert order == ["a", "b"]
      \end{minted}
    \end{column}
    \begin{column}{0.58\textwidth}
      \small
      \begin{minted}{python}
        # radionets tests/conftest.py
        import shutil

        import pytest


        @pytest.fixture(autouse=True, scope="session")
        def test_suite_cleanup_thing():
            yield

            build = "./tests/build/"
            print("Cleaning up tests.")

            shutil.rmtree(build)
      \end{minted}
      \vspace{0.5cm}
      \normalsize
      \begin{itemize}
        \item [\to] \texttt{pytest} fixtures provide defined, reliable and consistent context for the tests
        \item [\to] Essentially code to be run before or after a test, \eg, to prepare objects, data, or files
      \end{itemize}
    \end{column}
  \end{columns}
\end{darkframe}

\begin{frame}{Testing: Good Practices}
  \textcolor{ccyan}{\textbf{Test-driven development:}}
  \begin{itemize}
      \item [\to] Make testing part of your development process
      \item [\to] Write tests \emph{before} implementing your code:
      \begin{enumerate}
        \item Specify what the code should do
        \item Write tests that test those specifications
        \item Implement the code
      \end{enumerate}
  \end{itemize}
  \vspace{1em}
  In reality this may not always be feasible, but\dots
  \begin{itemize}
    \item Always try to write tests for your code, especially for critical components
    \item You can always add tests at a later time, in a separate commit
    \item Always write tests when you found and fixed a bug to ensure it doesn't reappear
  \end{itemize}
\end{frame}

\begin{frame}{Further Reading: Testing}
  \begin{columns}[t,onlytextwidth]
    \begin{column}{0.48\textwidth}
        \begin{itemize}
          \setlength{\itemsep}{1em}
          \item \iref{https://indico.desy.de/event/43817/sessions/19708/attachments/97952/134999/nikolai_testing_pyopp_aachen_25.06.2025.pdf}{Nikolai Krug's PYOPP Talk}
          \item \iref{https://github.com/nikoladze/pyopp-pytest-tutorial}{Nikolai Krug's pytest Tutorial}
          \item \iref{https://docs.python.org/3/library/doctest.html}{doctest}
          \item \iref{https://docs.python.org/3/library/unittest.html}{unittest}
          \item \iref{https://docs.pytest.org/en/stable/}{pytest}
          \item \iref{https://coverage.readthedocs.io/en/7.9.2/}{Coverage.py}
          \item \iref{https://tox.wiki/en/4.28.1/}{tox}
        \end{itemize}
    \end{column}
    \begin{column}{0.48\textwidth}
      \begin{itemize}
        \setlength{\itemsep}{1em}
        \item \iref{https://nox.thea.codes/en/stable/}{Nox} and \iref{https://hynek.me/articles/why-i-like-nox/}{Why I Like Nox}
        \item \iref{https://pytest-xdist.readthedocs.io/}{pytest-xdist}
        \item \iref{https://pytest-regressions.readthedocs.io/}{pytest-regression}
        \item \iref{https://pytest-mock.readthedocs.io/}{pytest-mock}
        \item \iref{https://hypothesis.readthedocs.io/}{pytest-hypothesis}
        \item \iref{https://pytest-order.readthedocs.io/en/latest/}{pytest-order}
        \item \iref{https://henryiii.github.io/se-for-sci/content/week03/testing.html}{Intro To Testing} by Henry Schreiner
      \end{itemize}
    \end{column}
  \end{columns}
\end{frame}
