\secslide[\fontsize{1.3cm}{1.3cm}\selectfont]{Packaging: The Basics}


\begin{frame}{What Even is a Package?}
  \begin{description}[\texttt{Distribution Package}]
    \setlength{\itemsep}{1.5em}
    \item [\texttt{Import Package}] Any Python module that you can \emph{import} using the \mintinline{python}+import+
      statement.
    \item [\texttt{Namespace Package}] Packages that allow you to \emph{unify} two packages with the \emph{same} name.
    \item [\texttt{Distribution Package}] An archive containing a \emph{collection} of import packages combined with
      \emph{metadata} such as dependencies.
  \end{description}
  \vspace{1cm}
  \uncover<2>{
  \begin{center}
    \huge\textcolor{ccyan}{When people talk about packages, they usually mean \textbf{distribution packages.}}
  \end{center}
  }
\end{frame}

\begin{darkframe}{How Does Python Find Installed Packages?}
  \begin{center}
  \huge\textcolor{ccyan}{Example: NumPy}
  \end{center}
  \vspace{0.5cm}
  \begin{minted}{shell-session}
    $ python -c "import numpy; print(numpy.__path__[0])"
    /home/anno/.local/conda/envs/pyopp_recap/lib/python3.12/site-packages/numpy

    $ ls -C $(python -c "import numpy; print(numpy.__path__[0])") | sort
    _array_api_info.py     doc		      __init__.py	 py.typed
    _array_api_info.pyi    dtypes.py	      __init__.pyi	 random
    char		      dtypes.pyi	      lib		 rec
    __config__.py	      exceptions.py	      linalg		 strings
    __config__.pyi	      exceptions.pyi	      ma		 testing
    _configtool.py	      _expired_attrs_2_0.py   matlib.py	 tests
    _configtool.pyi        _expired_attrs_2_0.pyi  matlib.pyi	 _typing
    conftest.py	      f2py		      matrixlib	 typing
    _core		      fft		      polynomial	 _utils
    core		      _globals.py	      __pycache__	 version.py
    ctypeslib	      _globals.pyi	      _pyinstaller	 version.pyi
    _distributor_init.pyi  __init__.pxd	      _pytesttester.pyi
    _distributor_init.py   __init__.cython-30.pxd  _pytesttester.py
  \end{minted}
\end{darkframe}


\begin{frame}[fragile]{How Do I Create a Package?}
  \begin{center}
    \huge\textcolor{ccyan!90!cblack}{\textbf{There is not \enquote{just one way} to create packages, but\dots}}
  \end{center}
  \vspace{1em}
  \begin{itemize}
    \setlength{\itemsep}{1em}
    \item Modern packaging uses a scaffolding called \texttt{pyproject.toml} with three important sections:
    \begin{description}[\texttt{[build-system]}]
      \item [\mintinline{toml}{[build-system]}] Allows you to describe what build backend to use.
      \item [\mintinline{toml}{[project]}] Sets up metadata for the package, such as the name or version.
      \item [\mintinline{toml}{[tool]}] A section for tool configuration.
    \end{description}
    \item An easy way to set up that scaffolding: \href{https://hatch.pypa.io/latest/}{{\footnotesize{\faExternalLink*}}\,\texttt{hatch}}
      \begin{minted}{shell-session}
        $ uv pip install hatch
        $ mamba install hatch
      \end{minted}
  \end{itemize}
\end{frame}

\begin{splitframe}[fragile]{How Do I Create a Package?}{Output}
  \begin{columns}[t,onlytextwidth]
    \begin{column}{0.58\textwidth}
      \begin{itemize}
        \item Use \textcolor{cpink}{\texttt{hatch}}'s CLI tool to quickstart creating a package:
          \begin{minted}{shell-session}
            $ hatch new my_package
          \end{minted}
      \end{itemize}
    \end{column}
    \hfill
    \begin{column}{0.38\textwidth}
      \begin{minted}{toml}
        my_package
        ├── src
        │   └── my_package
        │       ├── __about__.py
        │       └── __init__.py
        ├── tests
        │   └── __init__.py
        ├── LICENSE.txt
        ├── README.md
        └── pyproject.toml
      \end{minted}
    \end{column}
  \end{columns}
  \vspace{1em}
  \begin{columns}[t,onlytextwidth]
    \begin{column}{0.58\textwidth}
      \begin{itemize}
        \setlength{\itemsep}{1em}
        \item Let's see what this created:
          \begin{minted}{shell-session}
            $ head my_package/pyproject.toml
          \end{minted}
        \item You can also upgrade an existing project to use \texttt{hatch}:
          \begin{minted}{shell-session}
            $ hatch new --init
          \end{minted}
        \item Have a look at the \href{https://packaging.python.org/en/latest/guides/writing-pyproject-toml/}{{\footnotesize{\faExternalLink*}}\,Writing your \texttt{pyproject.toml}}
          guide to learn how to customise the \texttt{pyproject.toml} file
      \end{itemize}
    \end{column}
    \begin{column}{0.38\textwidth}
      \begin{minted}{toml}
        [build-system]
        requires = ["hatchling"]
        build-backend = "hatchling.build"

        [project]
        name = "my_package"
        dynamic = ["version"]
        description = ''
        readme = "README.md"
        requires-python = ">=3.8"
      \end{minted}
    \end{column}
  \end{columns}
\end{splitframe}

\begin{splitframe}[fragile]{Dependencies}{Example}
  \begin{columns}[t,onlytextwidth]
    \begin{column}{0.58\textwidth}
      \begin{itemize}
        \setlength{\itemsep}{1em}
        \item Dependencies for your project are defined with the \texttt{dependencies}
          key inside the \texttt{[project]} section
        \item You can set \href{https://packaging.python.org/en/latest/specifications/dependency-specifiers/}{{\footnotesize{\faExternalLink*}}\,dependency specifiers}
          (aka constraints) such as versions
      \end{itemize}
    \end{column}
    \hfill
    \begin{column}{0.38\textwidth}
      \begin{minted}{toml}
        [project]
        dependencies = [
          "numpy",
          "astropy<=6.1.0",
          "tomli;python_version<'3.11'",
        ]
      \end{minted}
    \end{column}
  \end{columns}
  \vspace{1em}
  \begin{columns}[t,onlytextwidth]
    \begin{column}{0.58\textwidth}
      \begin{itemize}
        \setlength{\itemsep}{1em}
        \item Define your optional dependencies in the \texttt{[project.optional-dependencies]} section
          and group them
        \item Install optional dependencies using
          {
          \usemintedstyle{code-light}
          \begin{minted}{shell-session}
            $ uv pip install my_package[plot]
            $ uv pip install "my_package[plot]"
          \end{minted}
          }
      \end{itemize}
    \end{column}
    \hfill
    \begin{column}{0.38\textwidth}
      \begin{minted}{toml}
        [project.optional-dependencies]
        plot = ["matplotlib"]
      \end{minted}
    \end{column}
  \end{columns}
  \vspace{1em}
\end{splitframe}

\begin{splitframe}[fragile]{Dependency Groups}{Example}
  \begin{columns}[t,onlytextwidth]
    \begin{column}{0.58\textwidth}
      \begin{itemize}
        \setlength{\itemsep}{1em}
        \item Fairly new (accepted \texttt{2024-10-10}): \href{https://peps.python.org/pep-0735/}{{\footnotesize{\faExternalLink*}}\,\texttt{PEP 735}} Dependency Groups
        \item Optional dependencies that are \emph{not} installed when a \emph{user} installs the package, \eg, via PyPI
          \begin{itemize}
            \item [\to] Prevent users from installing dev tools
          \end{itemize}
        \item Install the groups from within your source repo:
          \begin{minted}{shell-session}
            $ uv pip install --group dev
          \end{minted}
      \end{itemize}
    \end{column}
    \hfill
    \begin{column}{0.38\textwidth}
      \begin{minted}{toml}
        [dependency-groups]
        tests = ["pytest", "pytest-cov"]
        docs = ["sphinx"]
        dev = [
          "jupyter",
          "pre-commit",
          {include-group = "tests"},
          {include-group = "docs"},
        ]
      \end{minted}
    \end{column}
  \end{columns}

\end{splitframe}

\begin{splitframe}[fragile]{CLI Scripts}{Example}
  \begin{columns}[t,onlytextwidth]
    \begin{column}{0.58\textwidth}
      \begin{itemize}
        \setlength{\itemsep}{1em}
        \item We can expose scripts in our package using the \texttt{pyproject.toml}
          \texttt{[project.scripts]} section
        \item Similarly: Entry points, that allow the creation of plugins, and cross-platform
          compatibility
          \begin{itemize}
            \item [\textcolor{cpink}{\to}] See \href{https://setuptools.pypa.io/en/latest/userguide/entry_point.html}{{\footnotesize{\faExternalLink*}}\,Entry Points}
          \end{itemize}
      \end{itemize}
    \end{column}
    \hfill
    \begin{column}{0.38\textwidth}
      \begin{minted}{toml}
        src/my_package/cli.py:
      \end{minted}
      \vspace{0.25em}
      \begin{minted}{python}
        def print_message():
            print("Hello World!")
            raise SystemExit(1)
      \end{minted}
      \vspace{1em}
      \begin{minted}{toml}
        pyproject.toml:
      \end{minted}
      \vspace{0.25em}
      \begin{minted}{toml}
        [project.scripts]
        hello-world = "my_package.cli:print_message"
      \end{minted}
      \vspace{1em}
      \begin{center}
        \huge\textcolor{cpink}{\texttt{\textbf{Result}}}
      \end{center}
      \begin{minted}[escapeinside=||]{toml}
        |\textbf{\textcolor{cpink}{\$}}| hello-world
        Hello World!
      \end{minted}
    \end{column}
  \end{columns}
\end{splitframe}


\begin{frame}{Further Reading: Packaging}
  \begin{itemize}
    \setlength{\itemsep}{1em}
    \item \iref{https://agoose77.github.io/packaging-pyopp-2025/}{Packaging in Python (Angus Hollands)}
    \item \iref{https://packaging.python.org/en/latest/}{Python Packaging User Guide}
    \item \iref{https://www.pypa.io/en/latest/}{Python Packaging Authority}
    \item \iref{https://hatch.pypa.io/latest/}{hatch}
    \item \iref{https://docs.astral.sh/uv/}{uv}
    \item \iref{https://mamba.readthedocs.io/en/latest/}{mamba}
  \end{itemize}
\end{frame}
