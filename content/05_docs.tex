\secslide{Documentation}

\begin{frame}{Why Should We Document Our Code?}
  \begin{center}
    \huge\textcolor{ccyan!90!cblack}{Well documented code improves\dots}
  \end{center}
  \begin{itemize}
    \item Maintainability: Future developers, debugging, \dots
    \item Accessibility: Make your package easier to understand for new users
    \item Collaboration: Docs as a shared knowledge source
  \end{itemize}
\end{frame}

\begin{frame}[fragile]{Tool Of Choice: Sphinx}
  \begin{columns}[t, onlytextwidth]
    \begin{column}{0.68\textwidth}
      \begin{itemize}
        \setlength{\itemsep}{1em}
        \item FOSS, extensible documentation generator written in Python
        \item Multiple output formats: \texttt{HTML}, \LaTeX, ePub, and more\dots
        \item Content is written using a mark-up language (\texttt{reST} or \texttt{MyST})
        \item Support for various docstring formats (some through extensions)
        \item Install via uv or mamba:
          \begin{minted}{text}
            $ uv pip install sphinx
            $ mamba install sphinx
          \end{minted}
      \end{itemize}
    \end{column}
    \hfill
    \begin{column}{0.38\textwidth}
      \begin{center}
      \includegraphics[width=0.8\textwidth]{logos/sphinx-logo.pdf}
      \end{center}
    \end{column}
  \end{columns}
\end{frame}


\begin{frame}[fragile]{Getting Started}
  \begin{minted}[escapeinside=||]{shell-session}
    $ sphinx-quickstart docs
    |\textcolor{ccyan}{> Separate source and build directories (y/n) [n]:}| y
    |\textcolor{ccyan}{> Project name:}| ...
    |\textcolor{ccyan}{> Author name(s):}| ...
    |\textcolor{ccyan}{> Project release []:}| ...
    |\textcolor{ccyan}{> Project language [en]:}| ...
  \end{minted}

  \begin{center}
  \pause
  \begin{minipage}{.45\textwidth}
  \adjustbox{max height=5.5cm}{%
    \begin{forest}
      for tree={dir tree}
      [docs, opened
        [build, closed]
        [source, opened
          [\_static, closed]
          [\_templates, closed]
          [conf.py, pythonfile]
          [index.rst, textfile]
        ]
        [make.bat, batchfile]
        [Makefile, makefile]
      ]
    \end{forest}
  }
  \end{minipage}
  \hfill
  \pause
  \begin{minipage}{.45\textwidth}
    \adjustbox{max height=5.5cm}{%
      \begin{forest}
        for tree={dir tree}
        [docs, opened
          [\_build, closed]
          [\_static, closed]
          [\_templates, closed]
          [conf.py, pythonfile]
          [index.rst, textfile]
          [make.bat, batchfile]
          [Makefile, makefile]
        ]
      \end{forest}
    }
  \end{minipage}
  \end{center}
\end{frame}

{
\setbeamercolor{description item}{fg=cblack}
\begin{frame}[fragile]{Breakdown of the Generated Structure}
  \begin{description}[labelwidth=\widthof{\faFolderOpen \texttt{\_templates}}]
    \setlength{\itemindent}{-4em}
    \item [\textcolor{dircolor}{\faFolderOpen} \texttt{build}:] Output directory for the docs.
    \item [\textcolor{dircolor}{\faFolderOpen} \texttt{\_static}:] Directory for static elements such as images, icons, or logos.
    \item [\textcolor{dircolor}{\faFolderOpen} \texttt{\_templates}:] Used to store \iref{https://jinja.palletsprojects.com/en/stable/}{\texttt{Jinja}}
      templates for HTML page generation. %Also used by some Sphinx extensions.
    \item [\faFile* \texttt{index.rst}:] Root document; contains the root of the table of contents tree.
      % Effectively your landing page in the HTML version.
    \item [\faPython \texttt{conf.py}:] Main configuration file written in Python.
  \end{description}
\end{frame}
}

{
\usemintedstyle{code-light}
\begin{frame}[fragile]{Let's Build Our Docs}
  We will use the \texttt{Makefile} generated by \mintinline{shell-session}+sphinx-quickstart+ to build any format:
  \begin{minted}{shell-session}
    $ make <format>
  \end{minted}
  So, for the HTML version:
  \begin{minted}{shell-session}
    $ make html
  \end{minted}
  This will generate the HTML files for our docs inside the \texttt{build} directory.
  We can view the docs locally by running a Python HTTP server (in this case from inside the \texttt{docs} directory):
  \begin{minted}{shell-session}
    $ python -m http.server -d build/html [port]
  \end{minted}

  \begin{block}{Note}
    \mintinline{shell-session}+[port]+ is optional, see \mintinline{shell-session}+python -m http.server --help+.
  \end{block}
\end{frame}
}
