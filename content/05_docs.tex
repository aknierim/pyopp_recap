\secslide{Documentation}

\begin{frame}{Why Should We Document Our Code?}
  \begin{center}
    \huge\textcolor{ccyan!90!cblack}{Well documented code improves\dots}
  \end{center}
  \begin{itemize}
    \item Maintainability: Future developers, debugging, \dots
    \item Accessibility: Make your package easier to understand for new users
    \item Collaboration: Docs as a shared knowledge source
  \end{itemize}
\end{frame}

\begin{frame}[fragile]{Tool Of Choice: Sphinx}
  \begin{columns}[t, onlytextwidth]
    \begin{column}{0.68\textwidth}
      \begin{itemize}
        \setlength{\itemsep}{1em}
        \item FOSS, extensible documentation generator written in Python
        \item Multiple output formats: \texttt{HTML}, \LaTeX, ePub, and more\dots
        \item Content is written using a mark-up language (\texttt{reST} or \texttt{MyST})
        \item Support for various docstring formats (some through extensions)
        \item Install via uv or mamba:
          \begin{minted}{text}
            $ uv pip install sphinx
            $ mamba install sphinx
          \end{minted}
      \end{itemize}
    \end{column}
    \hfill
    \begin{column}{0.38\textwidth}
      \begin{center}
      \includegraphics[width=0.8\textwidth]{logos/sphinx-logo.pdf}
      \end{center}
    \end{column}
  \end{columns}
\end{frame}


{
\usemintedstyle{code-light}
\begin{frame}[fragile]{Getting Started}
  \begin{minted}[escapeinside=||]{shell-session}
    $ sphinx-quickstart docs
    |\textcolor{ccyan}{> Separate source and build directories (y/n) [n]:}| y
    |\textcolor{ccyan}{> Project name:}| ...
    |\textcolor{ccyan}{> Author name(s):}| ...
    |\textcolor{ccyan}{> Project release []:}| ...
    |\textcolor{ccyan}{> Project language [en]:}| ...
  \end{minted}

  \begin{center}
  \pause
  \begin{minipage}{.45\textwidth}
  \adjustbox{max height=5.5cm}{%
    \begin{forest}
      for tree={dir tree}
      [docs, opened
        [build, closed]
        [source, opened
          [\_static, closed]
          [\_templates, closed]
          [conf.py, pythonfile]
          [index.rst, textfile]
        ]
        [make.bat, batchfile]
        [Makefile, makefile]
      ]
    \end{forest}
  }
  \end{minipage}
  \hfill
  \pause
  \begin{minipage}{.45\textwidth}
    \adjustbox{max height=5.5cm}{%
      \begin{forest}
        for tree={dir tree}
        [docs, opened
          [\_build, closed]
          [\_static, closed]
          [\_templates, closed]
          [conf.py, pythonfile]
          [index.rst, textfile]
          [make.bat, batchfile]
          [Makefile, makefile]
        ]
      \end{forest}
    }
  \end{minipage}
  \end{center}
\end{frame}
}

{
\setbeamercolor{description item}{fg=cblack}
\begin{frame}[fragile]{Breakdown of the Generated Structure}
  \begin{description}[labelwidth=\widthof{\faFolderOpen \texttt{\_templates}}]
    \setlength{\itemindent}{-4em}
    \item [\textcolor{dircolor}{\faFolderOpen} \texttt{build}:] Output directory for the docs.
    \item [\textcolor{dircolor}{\faFolderOpen} \texttt{\_static}:] Directory for static elements such as images, icons, or logos.
    \item [\textcolor{dircolor}{\faFolderOpen} \texttt{\_templates}:] Used to store \iref{https://jinja.palletsprojects.com/en/stable/}{\texttt{Jinja}}
      templates for HTML page generation. %Also used by some Sphinx extensions.
    \item [\faFile* \texttt{index.rst}:] Root document; contains the root of the table of contents tree.
      % Effectively your landing page in the HTML version.
    \item [\faPython \texttt{conf.py}:] Main configuration file written in Python.
  \end{description}
\end{frame}
}

{
\usemintedstyle{code-light}
\begin{frame}[fragile]{Let's Build Our Docs}
  We will use the \texttt{Makefile} generated by \mintinline{shell-session}+sphinx-quickstart+ to build any format:
  \begin{minted}{shell-session}
    $ make <format>
  \end{minted}
  So, for the HTML version:
  \begin{minted}{shell-session}
    $ make html
  \end{minted}
  This will generate the HTML files for our docs inside the \texttt{build} directory.
  We can view the docs locally by running a Python HTTP server (in this case from inside the \texttt{docs} directory):
  \begin{minted}{shell-session}
    $ python -m http.server -d build/html [port]
  \end{minted}

  \begin{block}{Note}
    \mintinline{shell-session}+[port]+ is optional, see \mintinline{shell-session}+python -m http.server --help+.
  \end{block}
\end{frame}
}

\begin{darkframe}[fragile]{Setting Up conf.py}
  The \texttt{conf.py} file generated by Sphinx should look something like this:
  \begin{minted}{python}
    # -- Project information ------------------------
    project = 'pyopp'
    copyright = '2025, Author'
    author = 'Author'
    release = 'v0.1'

    # -- General configuration ----------------------
    extensions = []

    templates_path = ['_templates']
    exclude_patterns = []

    # -- Options for HTML output --------------------
    html_theme = 'alabaster'
    html_static_path = ['_static']
  \end{minted}
\end{darkframe}


\begin{darkframe}[fragile]{Setting Up conf.py | Project Information}
  \vspace*{0.25cm}
  Let's get some metadata from \texttt{pyproject.toml} using \texttt{tomli} or \texttt{tomllib} (Python $\geqslant$ \texttt{3.11}):\\[0.25\baselineskip]
  \footnotesize
  \begin{minted}{python}
    #!/usr/bin/env python3
    import datetime
    import sys
    from pathlib import Path

    import package                                                           # your package

    if sys.version_info < (3, 11):
        import tomli as tomllib
    else:
        import tomllib

    pyproject_path = Path(__file__).parent.parent.parent / "pyproject.toml"  # Get path of pyproject.toml
    pyproject = tomllib.loads(pyproject_path.read_text())                    # Load contents

    project = pyproject["project"]["name"]                                   # Get project name
    author = pyproject["project"]["authors"][0]["name"]                      # Get author name
    copyright = "{}.  Last updated {}".format(
        author, datetime.datetime.now().strftime("%d %b %Y %H:%M")
    )                                                                        # Set copyright string
    python_requires = pyproject["project"]["requires-python"]                # Get minimum python version requirement
    rst_epilog = f"""
    .. |python_requires| replace:: {python_requires}
    """                                                                      # Make python_requires var accessible

    version = pyvisgen.__version__                                           # Get version
    release = version                                                        # Full release version
  \end{minted}
\end{darkframe}


\begin{darkframe}[fragile]{Setting Up conf.py | General Configuration}
  Sphinx extensions add functionality and customization. The following extensions
  are some of the extensions we always use in our docs:\\[0.25\baselineskip]
    \small
    \begin{minted}{python}
      extensions = [
          "sphinx.ext.autodoc",                # Imports modules and pulls in documentation from docstrings
          "sphinx.ext.intersphinx",            # Cross-references to other projects
          "sphinx.ext.coverage",               # Collects doc coverage stats
          "sphinx.ext.viewcode",               # Links to highlighted source code (i.e. "[source]" button)
          "sphinx_automodapi.automodapi",                    # Automatically generates module documentation
          "sphinx_automodapi.smart_resolver",                # Helps resolving some imports
          "numpydoc",                                        # Support for the NumPy docstring format
          "IPython.sphinxext.ipython_console_highlighting",  # Syntax highlighting of ipython prompts
          "sphinx_copybutton",                               # Adds a copybutton to code blocks
      ]
    \end{minted}
\end{darkframe}

{
\usemintedstyle{code-light}
\begin{frame}[fragile]{Setting Up \texttt{conf.py} | General Configuration}
    Some extensions are not shipped with Sphinx and need to be installed separately in your environment:
    \begin{minted}{shell-session}
      $ mamba install sphinx-automodapi numpydoc pydata-sphinx-theme sphinx-copybutton
    \end{minted}
    or with \texttt{uv}
    \begin{minted}{shell-session}
      $ uv pip install sphinx-automodapi numpydoc pydata-sphinx-theme sphinx-copybutton
    \end{minted}
\end{frame}
}

\begin{darkframe}[fragile]{Setting Up conf.py | General Configuration}
  Now we can set up some more settings for the extensions:\\[0.25\baselineskip]
    \begin{minted}{python}
    # gets rid of some errors during build
    numpydoc_show_class_members = False
    numpydoc_class_members_toctree = False

    intersphinx_mapping = {
       "numpy": ("https://numpy.org/doc/stable", None),
       ...
    }

    suppress_warnings = ["intersphinx.external"]  # sometimes necessary

    templates_path = ["_templates"]
    exclude_patterns = ["build", "Thumbs.db", ".DS_Store", "changes", "*.log"]

    source_suffix = {".rst": "restructuredtext"}  # Set .rst files as source files for docs
    master_doc = "index"                          # index.rst as root file
  \end{minted}
\end{darkframe}

\begin{darkframe}[fragile]{Setting Up conf.py | HTML And Theme Options}
  HTML options set the look of your docs. The Sphinx community has created a variety of
  themes you can choose from.\\[0.25\baselineskip]
  \begin{minted}{python}
    html_theme = "pydata_sphinx_theme"            # Modern, widely used theme

    html_static_path = ["_static"]
    html_favicon = "_static/favicon/favicon.ico"  # Icon file for browser tabs
    html_css_files = ["custom.css"]               # Custom CSS settings like colors or fonts
    html_file_suffix = ".html"

    html_theme_options = {...}                    # Depends on the theme

    html_title = f"{project}"                     # e.g. your project name
    htmlhelp_basename = project + " docs"
  \end{minted}
  \vspace{0.25\baselineskip}
  Check out \emph{Sphinx Themes Gallery} for a curated list of available themes:
  \iref{https://sphinx-themes.org/}{sphinx-themes.org}
\end{darkframe}
